Time Complexity is a concept in computer science used to describe how the runtime of an algorithm increases with the size of the input. It's usually expressed using Big O notation, such as O(1), O(n), O(log n), O(n²), etc.

Q 1. 🔍 What is Time Complexity?
Time complexity tells you:

How many operations an algorithm performs as the input size (n) grows.

✅ Why is Time Complexity Important?
Reason	Explanation
1. Predict Performance	Helps estimate how fast or slow your code will run for large input sizes.
2. Compare Algorithms	Allows you to compare two solutions and pick the one that scales better.
3. Avoid Slowness	Inefficient time complexity (like O(n²)) can cause your program to crash or freeze for large inputs.
4. Interview/Competitive Coding	A key concept in coding interviews and contests — correct logic alone is not enough if it's too slow.
5. Resource Optimization	Helps save CPU time, energy, and cost in real-world applications (like search engines, AI, etc.).

💡 Example:
Suppose you want to search for a number in a list:

Linear Search: Checks each element → Time Complexity: O(n)

Binary Search: Divides list in half each time → Time Complexity: O(log n) (much faster for large data)


Q 2. 📊 Types of Asymptotic Notations and Their Differences

Notation	Name	Used For	Meaning	Tight Bound?	Mathematical Definition	Example
O(f(n))	Big-O Notation	Worst-case	Time will not exceed this function	❌ No	T(n) ≤ c·f(n) for all n ≥ n₀	O(n²) → ≤ n²
Ω(f(n))	Big-Omega Notation	Best-case	Time will at least be this function	❌ No	T(n) ≥ c·f(n) for all n ≥ n₀	Ω(n) → ≥ n
Θ(f(n))	Theta Notation	Average/Exact-case	Time will grow exactly like this function	✅ Yes	c₁·f(n) ≤ T(n) ≤ c₂·f(n) for all n ≥ n₀	Θ(n) → exactly n
o(f(n))	Little-o Notation	Strict upper bound	Time grows strictly less than f(n)	❌ No	T(n) < c·f(n) ∀ c > 0, as n → ∞	o(n²) → < n²
ω(f(n))	Little-omega Notation	Strict lower bound	Time grows strictly more than f(n)	❌ No	T(n) > c·f(n) ∀ c > 0, as n → ∞	ω(n) → > n

📌 Key Differences:
Big-O (O): Upper bound → How slow it can get at worst.

Big-Omega (Ω): Lower bound → How fast it can be at best.

Theta (Θ): Tight bound → Exact behavior (both upper and lower).

Little-o (o): Strict upper → Faster than a specific bound.

Little-omega (ω): Strict lower → Slower than a specific bound.

💡 Visual Hierarchy:
markdown
Copy
Edit
                 Θ(f(n))
                /       \
             Ω(f(n))   O(f(n))
              |           |
           ω(f(n))     o(f(n))
🚀 Real-Life Analogy (Speed of a Car):
Notation	Analogy
O(n)	"I’ll take at most 2 hours to reach."
Ω(n)	"I’ll take at least 2 hours to reach."
Θ(n)	"I’ll take exactly 2 hours every time."
o(n)	"I’ll never take full 2 hours — always less."
ω(n)	"I’ll always take more than 2 hours."






Q1. What is Time Complexity?
 Time complexity measures how the execution time of an algorithm increases with the input size `n`.
 Example:
 If an algorithm takes 5 seconds for 100 items and 10 seconds for 200 items, it's linear time: O(n).
 Real-life example:
 Imagine scanning a list of names on paper. If the list has 100 names, you take 1 minute. For 200
 names, it takes 2 minutes. This shows linear time complexity: O(n).


 Q2. What are the types of Asymptotic Notations? Differentiate them.
 Asymptotic notations help describe the time complexity of algorithms in different scenarios.
 Notation Comparison Table:
 | Notation | Case Type     | Bound Type | Description | Example |
 |----------|---------------|----------------|------------------------------------------|----------|
 | O(f(n)) | Worst-case    | Upper Bound | Max time it can take | O(n²) |
 | Ω(f(n)) | Best-case     | Lower Bound | Min time it will take | Ω(n) |
 | Θ(f(n)) | Average/Exact | Tight Bound | Exact growth rate | Θ(n log n)|
 | o(f(n)) | -| Strict Upper | Always faster than | o(n²) |
 | ω(f(n)) | -| Strict Lower | Always slower than | ω(n) |
 Real-life example:- O(n): You may check every document in a file cabinet.- Ω(1): You might find what you need in the first drawer.- Θ(n): On average, you check half the drawers.


 Q3. Explain Best, Average, and Worst Case with Example.
 These cases help understand how an algorithm performs under different input conditions.
 Best Case:- Most favorable input.- Takes the least time.- Example: In Linear Search, the target is the first element → Ω(1).
 Average Case:- Input is random.- Represents typical behavior.- Example: Target is somewhere in the middle → Θ(n).
 Worst Case:- Least favorable input.- Takes the most time.- Example: Target is last or not found at all → O(n).
 Real-life example:
 Searching a word in a dictionary:- Best case: Found on the first page.- Average case: Found in the middle.- Worst case: Found on the last page or not at all.


Q4. How do iterative and recursive solutions compare in terms of time and space
 complexity?
 Iterative and recursive solutions can solve the same problem but differ in how they use memory and
 sometimes in execution time.
 Time Complexity:- Both iterative and recursive solutions can have the same time complexity depending on the
 problem.- Example: Factorial (iterative vs recursive) both have O(n) time.
 Space Complexity:- Iterative: Uses constant space O(1), because it uses loops.- Recursive: Uses additional space O(n) due to function call stack.
 Conclusion:- Iteration is usually more space-efficient.- Recursion is more elegant and suitable for divide-and-conquer problems (like Merge Sort).


 Q5. How does amortized analysis differ from average case analysis?
 Amortized Analysis:- Describes the average time per operation over a sequence of operations, **even if a single
 operation is costly**.- Used when occasional operations are expensive but rare.- Example: Array resizing in dynamic array (like ArrayList).
 Average Case Analysis:- Describes the average time taken over all possible inputs of a given size.- Based on probability distribution of inputs.- Example: Average case of Linear Search is O(n/2).
 Difference:- Amortized looks at cost over time (sequence of operations).- Average looks at expected cost over all possible inputs.


 Q6. Write a program to compute the time complexity of nested loops.
 for (int i = 0; i < n; i++) {
 for (int j = 0; j < n; j++) {
 // Constant time operation
 }
 }
 // Time Complexity: O(n²)


 Q7. Implement a function that performs linear search and counts steps.
 def linear_search(arr, target):
 steps = 0
 for i in range(len(arr)):
 steps += 1
 if arr[i] == target:
 return i, steps
 return -1, steps


 Q8. Simulate recursion and count stack depth using a custom counter.
def simulate_recursion(n, depth=0):
 if n == 0:
 return depth
 return simulate_recursion(n - 1, depth + 1)
 # simulate_recursion(5) will return 5 as the stack depth


 Q9. Compare Bubble Sort and Merge Sort (time and space).
 Bubble Sort:- Time: O(n²)- Space: O(1)
 Merge Sort:- Time: O(n log n)- Space: O(n)
 Merge Sort is much faster on large data and uses extra space for merging.


 Q10. What is Linear Search?
 Linear Search checks each element one by one until it finds the target or reaches the end.
 Time Complexity: O(n)
 Used on unsorted data.


 Q11. What is the prerequisite for Binary Search to work?
 The array or list must be **sorted** in ascending or descending order.


 Q12. Why is Binary Search more efficient than Linear Search?
 Binary Search eliminates half of the search space in each step.- Time Complexity: O(log n)- Linear Search checks each element: O(n)


 Q13. Can Binary Search be used in Linked List?
 No, because linked lists do not allow random access in O(1) time.
 To reach the middle, we must traverse nodes = O(n)


 Q14. Implement Linear Search for list of names.
 def search_name(names, target):
 for i in range(len(names)):
 if names[i] == target:
 return i
 return -1


 Q15. Implement Recursive and Iterative Binary Search.
 # Iterative
 def binary_search_iterative(arr, target):
 left, right = 0, len(arr) - 1
 while left <= right:
 mid = (left + right) // 2
if arr[mid] == target:
 return mid
 elif arr[mid] < target:
 left = mid + 1
 else:
 right = mid - 1
 return -1
 # Recursive
 def binary_search_recursive(arr, target, left, right):
 if left > right:
 return -1
 mid = (left + right) // 2
 if arr[mid] == target:
 return mid
 elif arr[mid] < target:
 return binary_search_recursive(arr, target, mid + 1, right)
 else:
 return binary_search_recursive(arr, target, left, mid - 1)


 
 Q16. Binary Search for first and last occurrence of a number.
 def first_occurrence(arr, target):
 res = -1
 left, right = 0, len(arr) - 1
 while left <= right:
 mid = (left + right) // 2
 if arr[mid] == target:
 res = mid
 right = mid - 1
 elif arr[mid] < target:
 left = mid + 1
 else:
 right = mid - 1
 return res
 def last_occurrence(arr, target):
 res = -1
 left, right = 0, len(arr) - 1
 while left <= right:
 mid = (left + right) // 2
 if arr[mid] == target:
 res = mid
 left = mid + 1
 elif arr[mid] < target:
 left = mid + 1
 else:
 right = mid - 1
 return res


 Q17. How does Bubble Sort work?
 Bubble Sort repeatedly swaps adjacent elements if they are in the wrong order.
 Each pass 'bubbles' the largest element to the end.
Time Complexity: O(n²)
 Space: O(1)


 Q18. Time Complexity of Merge Sort?
 Merge Sort divides the array into halves and merges them after sorting.
 Time: O(n log n)
 Space: O(n)


 Q19. How is recursion used in Merge Sort?
 Merge Sort uses recursion to divide the array into smaller parts until one element remains, then
 merges them in sorted order.
 Example:
 merge_sort([5, 2, 4, 1]) → calls merge_sort([5,2]) and merge_sort([4,1]) recursively



 Q20. Implement Merge Sort recursively.
 def merge_sort(arr):
 if len(arr) <= 1:
 return arr
 mid = len(arr) // 2
 left = merge_sort(arr[:mid])
 right = merge_sort(arr[mid:])
 return merge(left, right)
 def merge(left, right):
 result = []
 i = j = 0
 while i < len(left) and j < len(right):
 if left[i] < right[j]:
 result.append(left[i])
 i += 1
 else:
 result.append(right[j])
 j += 1
 result.extend(left[i:])
 result.extend(right[j:])
 return result



 Q21. Compare Merge Sort with Bubble Sort by counting steps.
 Add a step counter inside both merge and bubble sort loops.
 Compare total steps for small vs large arrays.
 Bubble Sort will take much more steps as data grows (O(n²) vs O(n log n)).



 Q22. Visualize Bubble and Merge Sort operations.
 Use libraries like matplotlib or Pygame to show sorting steps visually.
 Each step can draw bars for elements and swap animations to understand the process